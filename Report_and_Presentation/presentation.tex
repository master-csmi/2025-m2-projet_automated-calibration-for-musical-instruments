% presentation.tex
% Beamer template (français) pour une présentation de projet (structure en 4 parties)
\documentclass[11pt]{beamer}

% Langue et encodage
\usepackage[T1]{fontenc}
\usepackage[utf8]{inputenc}
\usepackage[french]{babel}
\usepackage{lmodern}
\usepackage{microtype}
\usepackage{graphicx}
\usepackage{amsmath,empheq}
\usepackage{hyperref}
\usepackage[backend=biber, style=numeric]{biblatex}
\addbibresource{biblio.bib}
\hypersetup{pdfauthor={R. Vivant}, pdftitle={Automated Calibration of Physical Models}}

% Thème et couleurs
\usetheme{Madrid}            % thème clair et professionnel
\usecolortheme{seagull}
\setbeamertemplate{navigation symbols}{} % supprime icônes de navigation
% Avoid requesting small caps in sans fonts (Latin Modern Sans doesn't provide sc);
% force upright (no small caps) for title/frametitle/author to prevent font shape warnings.
\setbeamerfont{frametitle}{series=\bfseries,shape=\upshape}
\setbeamerfont{title}{series=\bfseries,shape=\upshape}
\setbeamerfont{author}{series=\mdseries,shape=\upshape}

% Couleurs personnalisées
\definecolor{accent}{RGB}{0,102,153}
\setbeamercolor{frametitle}{bg=accent, fg=white}
\setbeamercolor{title}{bg=white, fg=accent}
\setbeamercolor{structure}{fg=accent}

% Infos du titre
\title[Automated Calibration]{Automated Calibration of Physical Models for Musical Instruments}
\author[R. M. Vivant]{Rodolphe Milton Vivant}
\institute[Ufr Mathématiques et Informatique de Strasbourg]{Master 2 — Projet}
\date{\today}

% En-tête / pied de page
\setbeamertemplate{footline}{
    \leavevmode%
    \hbox{%
    \begin{beamercolorbox}[wd=.65\paperwidth,ht=2.5ex,dp=1ex,left]{author in head/foot}%
        \hspace{1em}\insertshortauthor{} — \insertshortinstitute
    \end{beamercolorbox}%
    \begin{beamercolorbox}[wd=.35\paperwidth,ht=2.5ex,dp=1ex,right]{date in head/foot}%
        \insertshortdate{}\hspace{1em}
    \end{beamercolorbox}}%
    \vskip0pt%
}
\setbeamertemplate{headline}{}

% Commandes utiles
\newcommand{\slideTitle}[1]{\begin{frame}\frametitle{#1}}

% Document
\begin{document}

% Page de titre
\begin{frame}[plain]
    \titlepage
    % Trois logos sous la page de titre — adaptez les chemins/noms de fichiers
    \vfill
    \begin{center}
        \includegraphics[height=2.5cm]{Images/inria_logo.png}\hspace{0.0em}%
        \includegraphics[height=2.5cm]{Images/modartt_logo.png}\hspace{0.0em}%
        \includegraphics[height=1.6cm]{Images/logo_irma.png}
    \end{center}
    \vspace{0.8em}
    % Si vous avez un logo : \includegraphics[height=1.5cm]{logo.png}
\end{frame}

% Plan
\begin{frame}
    \frametitle{Plan}
    \tableofcontents
\end{frame}

% Section 1
\section{Objectives and Context}
\slideTitle{Context}
\begin{itemize}
    \item Automatically adjust parameters of physical models to match target sounds of reed instruments.
    \item Joint collaboration between IRMA and Modartt.
    \item Modartt \cite{Modartt} is a French company specialized in the development of artistic and technological application software (PIANOTEQ model).
\end{itemize}
\end{frame}

\slideTitle{Objectives}
In JAX framework:
\begin{itemize}
    \item Implement a discontinuous Galerkin discretisation of the mixt form of the wave and numerically solve it.
    \item Implement the coupling with an ODE and the boundary conditions of a reed instrument.
    \item Compute the gradient with respect to all parameters.
\end{itemize}
\end{frame}

\section{Theoritical Aspects}
\slideTitle{Reed Instruments}
\begin{itemize}
    \item \cite{Universalis_Anche} Reed : thin, flexible lamina mounted at the mouthpiece or at the entrance of an air column resonator.
    \item \cite{chaigne:hal-05370840} When subjected to a pressure difference $\rightarrow$ generates self-sustained oscillations (sound).
\end{itemize}
            
\begin{columns}[T]
    \begin{column}{0.5\textwidth}
        \begin{itemize}
            \item Two principal categories of reed instruments:
                \begin{itemize}
                    \item Free reeds (harmonicas, accordions, ...).
                    \item Beating reeds (clarinets, saxophones, ...).
                \end{itemize}
        \end{itemize}
    \end{column}
    \begin{column}{0.5\textwidth}
        \centering
        \vspace{-1ex}
        \includegraphics[width=\linewidth,keepaspectratio]{Images/reed_column.png} 
        \par\smallskip
        \scriptsize Example of a saxophone reed and mouthpiece.
        \label{fig:sax_reed}
    \end{column}
\end{columns}
\end{frame}

\slideTitle{Physical Model \cite{MAMAMIA2024}}
\noindent\begin{minipage}[t]{\linewidth}
\scriptsize
\begin{subequations}
\begin{empheq}[left=\empheqlbrace]{align}
    \frac{S(x)}{c S^\ast}\,\partial_t p(x,t) + \partial_x v(x,t) &= 0 \label{eq:1a}\\[4pt]
    \frac{S^\ast}{c S(x)}\,\partial_t v(x,t) + \partial_x p(x,t) &= 0 \label{eq:1b}\\[4pt]
    \frac{Z^\ast}{T}\,\partial_t \phi + \sqrt{\alpha}\,p(L,t) &= 0 \label{eq:1c}\\[4pt]
    -\frac{\beta}{Z^\ast T}\,p(L,t) + v(L,t) + \sqrt{\alpha}\,\phi &= 0 \label{eq:1d}\\[4pt]
    -v(0,t) + \zeta_\ell(y)\,F(\gamma - p(0,t)) + \varepsilon\frac{\kappa}{\omega_r}\,\partial_t y &= 0 \label{eq:1e}\\[6pt]
    \frac{1}{\omega_r^2}\,\partial_{tt} y + \frac{1}{Q_r\omega_r}\,\partial_t y + y - 1 
    &= \varepsilon(\gamma - p(0,t)) \label{eq:1f}
\end{empheq}
\end{subequations}
\end{minipage}
{\scriptsize
\begin{itemize}
    \item Equations \eqref{eq:1a} and \eqref{eq:1b}: wave propagation (pressure and acoustic volume velocity)in a variable section tube.
    \item Equations \eqref{eq:1c} and \eqref{eq:1d}: Radiative phenomenons ($\phi$).
    \item Equations \eqref{eq:1e} and \eqref{eq:1f}: effects of the instrumental player at the input boundary.
\end{itemize}
}
\end{frame}

% Section 2
\section{Methods}
\slideTitle{Discontinuous Galerkin Method \cite{HesthavenWarburton2008}}
\begin{itemize}
    \item Numerical method to solve PDEs.
    \item Consider a scalar conservation law :
\begin{equation}\label{eq:transport}
\partial_t u(x,t) + \partial_x F(u(x,t)) = 0,\qquad u:\Omega\times[0,T]\to\mathbb R.
\end{equation}
    \begin{itemize}
        \item where $u \in \Omega \times [0,T] \rightarrow \mathbb{R}$,
        \item $F : \mathbb{R} \rightarrow \mathbb{R}$ is the flux function
    \end{itemize}
    \item Let's consider an Euler explicit time discretization with time step $\Delta t$.
    \item At each time step $t^n = n\Delta t$, we want to approximate $u(x,t^n)$ by $u^n(x)$.
    \item this gives :
\begin{equation}
    u^{n+1}(x) = u^n(x)  - \Delta t \, \partial_x F(u^n(x))
\end{equation}
\end{itemize}
\end{frame}

\slideTitle{Discontinuous Galerkin Method (contd.)}
Now we consider that the state $u^n$ is known: 
\begin{itemize}
    \item Partition $\Omega$ into $K$ non-overlapping elements $\{\Omega_k\}$.
    \item On each element approximate $u^n(x)\approx u_h^n(x)=\sum_j \alpha_{k,j}^n \phi_j(x)$ with local basis $\{\phi_j\}$ and enforce the weak (Galerkin) form:
\begin{equation}
    \int_{\Omega_k} u^{n+1}(x)\phi_i = \int_{\Omega_k} u^n(x) \phi_i - \Delta t \, \int_{\Omega_k}  \partial_x F(u^n(x)) \phi_i(x)\,dx
\end{equation}
    \item Integrate by parts:
  \scriptsize
  \begin{equation}
    \sum_j \alpha_{k,j}^{n+1}\!\int_{\Omega_k}\!\phi_j\phi_i
    =
    \int_{\Omega_k} u^n_h\phi_i
    +\Delta t\int_{\Omega_k} F(u^n_h)\partial_x\phi_i\,dx
    -\Delta t\;\left[ F(u^n(x)) \phi_i(x) \right]_{\partial \Omega_k},
    \end{equation}  
    \normalsize
    \item Denote $M_{ij}=\int_{\Omega_k}\phi_j\phi_i$. Then
  \begin{equation}
    M\boldsymbol{\alpha}_k^{\,n+1}=\mathbf b_k,
  \end{equation}
  where $\mathbf b_k$ is known from $u^n_h$. In practice $M$ is small and local.
    \item Note: choose $\Delta t$ to satisfy a CFL condition; impose boundary conditions via the numerical flux.
\end{itemize}
\end{frame}

\slideTitle{Simplified Model}
To validate the implementation of the DGM, we consider a simplified model :
\begin{itemize}
    \item System of equations :
    \begin{equation}
        \begin{cases}
            \partial_t p(x,t) + \partial_x v(x,t) = 0 \\
            \partial_t v(x,t) + \partial_x p(x,t) = 0
        \end{cases}
    \end{equation}
    \item Domain : $x \in [0,L]$, $t \in [0,T]$.
    \item Initial conditions : $p(x,0) = p_0(x)$, $v(x,0) = v_0(x)$.
    \item Riemann variables : $r(x,t) = p(x,t) + v(x,t)$, $s(x,t) = p(x,t) - v(x,t)$.
    \begin{equation}
        \begin{cases}
            \partial_t r(x,t) + \partial_x r(x,t) = 0 \\
            \partial_t s(x,t) - \partial_x s(x,t) = 0
        \end{cases}
    \end{equation}
\end{itemize}
\end{frame}

\slideTitle{Simplified Model (contd.)}
\begin{itemize}
    \item Exact solution :
    \begin{equation}
        \begin{cases}
            r(x,t) = r_0(x-t) = p_0(x-t) + v_0(x-t) \\
            s(x,t) = s_0(x+t) = p_0(x+t) - v_0(x+t)
        \end{cases}
    \end{equation}
    \item Therefore :
    \begin{equation}
        \begin{cases}
            \begin{aligned}
            p(x,t) &= \tfrac12\bigl[r_0(x-t) + s_0(x+t)\bigr] \\
            &= \tfrac12\bigl[p_0(x-t) + v_0(x-t) + p_0(x+t) - v_0(x+t)\bigr] \\
            v(x,t) &= \tfrac12\bigl[r_0(x-t) - s_0(x+t)\bigr] \\
            &= \tfrac12\bigl[p_0(x-t) + v_0(x-t) - p_0(x+t) + v_0(x+t)\bigr]
            \end{aligned}
        \end{cases}
    \end{equation}
\end{itemize}
\end{frame}

\subsection{DG's results}

\slideTitle{Result}
\begin{columns}[T]
    \begin{column}{0.58\textwidth}
        \centering
        \includegraphics[width=\linewidth]{Images/dg_solution.png}
        \scriptsize DG solution.
        \label{fig:dg_result}
    \end{column}
    \begin{column}{0.42\textwidth}
       \begin{itemize}
        \item Numerical solution with DGM.
        \item $p_0(x) = e^{-0.5((x-0.5)/\sigma)^2}$, $v_0(x) = 0$
        \item For $N=800$ cells, $L_2$ error is $\approx 9.1 \times 10^{-5}$
       \end{itemize} 
    \end{column}
\end{columns}
\end{frame}

\slideTitle{Convergence and Efficiency}
\begin{columns}[T]
    \begin{column}{0.5\textwidth}
        \centering
        \includegraphics[width=\linewidth]{Images/dg_convergence_euler.png}
        \scriptsize DG convergence with Explicit Euler scheme.
        \label{fig:dg_convergence_euler}
    \end{column}
    \begin{column}{0.5\textwidth}
        \centering
        \includegraphics[width=\linewidth]{Images/dg_convergence_rk2.png}
        \scriptsize DG convergence with Runge-Kutta 2nd order scheme.
        \label{fig:dg_convergence_rk2}
    \end{column}
\end{columns}
\end{frame}


\begin{frame}[allowframebreaks]{References}
\printbibliography
\end{frame}

\end{document}