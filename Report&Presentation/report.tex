% report.tex
% Scientific article template in English
% Recommended compilation: pdflatex report.tex -> biber report -> pdflatex report.tex (x2)

\documentclass[11pt,a4paper]{article}

% Encoding and language
\usepackage[T1]{fontenc}
\usepackage[utf8]{inputenc}
\usepackage[english]{babel}

% Page layout
\usepackage{geometry}
\geometry{margin=2.5cm}
\usepackage{microtype}

% Graphics, tables, mathematics
\usepackage{graphicx}
\usepackage{booktabs}
\usepackage{amsmath,amssymb,amsthm}
\usepackage{empheq}
\usepackage{siunitx}

% Links, citations
\usepackage[colorlinks=true,linkcolor=blue,citecolor=blue,urlcolor=blue]{hyperref}
\usepackage{csquotes}
% biblatex is loaded later with the desired options to avoid an "Option clash for package biblatex" error

\usepackage[
    backend=biber,
    style=numeric,   % <-- citations sous forme d'entiers
    sorting=none     % pour garder l'ordre d'apparition si tu veux
]{biblatex}

\addbibresource{biblio.bib}

% Useful environments
\theoremstyle{plain}
\newtheorem{theorem}{Theorem}[section]
\theoremstyle{definition}
\newtheorem{definition}{Definition}[section]

% Handy commands
\newcommand{\ie}{\textit{i.e.}}
\newcommand{\eg}{\textit{e.g.}}

% PDF metadata
\hypersetup{
    pdftitle={Automated Calibration of Physical Models for Musical Instruments},
    pdfauthor={Rodolphe Milton VIVANT},
    pdffitwindow=true
}

\title{Automated Calibration of Physical Models for Musical Instruments\\[0.5em]}

\author{Rodolphe Milton VIVANT\thanks{UFR de Mathématique et Informatique de Strasbourg, rodolphe.vivant3@etu-unistra.fr}}
\date{\today}

\begin{document}

\maketitle

\begin{abstract}
Abstract (in English): 
\end{abstract}



\newpage
\section{Introduction}
Present the context, a concise state of the art, the motivation and the main contributions of the paper.

During my project of the thrid semester of the CSMI Master degree, I worked on the automated calibration 
of physical models for musical instruments. 
The goal of this project was to develop a method to automatically adjust the parameters of physical models 
so that they accurately reproduce the sound characteristics of reed instruments. \newline
I worked under the supervision of Dr Emmanuel Franck, Dr Laurent Navoret and Dr Victor Michel-Dansac from 
the Institut de Recherche Mathématique Avancée (IRMA) in Strasbourg and Dr Juliette Chabassier for Modartt. \newline
Modartt \cite{Modartt} is a french company specialized in the development of artistic and technological application softwares.

\subsection{Reed Instruments}
A reed is a thin, flexible lamina mounted at the mouthpiece or at the entrance of an air column resonator that acts \cite{chaigne:hal-05370840} as a valve that controls the airflow driven by a pressure difference between the upstream region 
(the player’s mouth) and the downstream region (the mouthpiece or mouthpiece tip of the instrument). \newline
Two principal categories of reed instruments:
\begin{itemize}
    \item Free reeds (harmonicas, accordions, ...).
    \item Beating reeds (clarinets, saxophones, ...).
\end{itemize}
% Figure : schéma de l'anche / embouchure
\begin{figure}[htbp]
    \centering
    % Remplacez 'figs/reed.png' par le chemin et l'extension réels de votre image (pdf, png, jpg...)
    \includegraphics[width=0.9\textwidth]{Images/reed_column.png}
    \caption{Reed and saxophone.}
    \label{fig:reed_schematic}
\end{figure}
\subsection{State of the art}
The equations governing the behavior of reed instruments are mathematically described \cite{MAMAMIA2024} by a set of coupled 
differential equations (Partial Differential Equations - PDEs, Ordinary Differential Equations - ODEs, and 
algebraic equations). These equations capture the complex interactions between the pressure ($p$), the acoustic
volume velocity ($v$) \eqref{eq:1a} \eqref{eq:1b}, some radiative phenomenons ($\phi$) \eqref{eq:1c} \eqref{eq:1d}, 
and the effects of the instrumental player \eqref{eq:1e} \eqref{eq:1f} :
% Système d'équations décrivant un modèle réducteur pour un instrument à anche
\begin{subequations}
\begin{empheq}[left=\empheqlbrace]{align}
  \frac{S(x)}{c S^\ast}\,\partial_t p(x,t) + \partial_x v(x,t) &= 0 \label{eq:1a}\\[4pt]
  \frac{S^\ast}{c S(x)}\,\partial_t v(x,t) + \partial_x p(x,t) &= 0 \label{eq:1b}\\[4pt]
  \frac{Z^\ast}{T}\,\partial_t \phi + \sqrt{\alpha}\,p(L,t) &= 0 \label{eq:1c}\\[4pt]
  -\frac{\beta}{Z^\ast T}\,p(L,t) + v(L,t) + \sqrt{\alpha}\,\phi &= 0 \label{eq:1d}\\[4pt]
  -v(0,t) + \zeta_\ell(y)\,F(\gamma - p(0,t)) + \varepsilon\frac{\kappa}{\omega_r}\,\partial_t y &= 0 \label{eq:1e}\\[6pt]
  \frac{1}{\omega_r^2}\,\partial_{tt} y + \frac{1}{Q_r\omega_r}\,\partial_t y + y - 1 
  &= \varepsilon(\gamma - p(0,t)) \label{eq:1f}
\end{empheq}
\end{subequations}

where $S(x)$ is the cross-sectional area of the instrument at position $x$, $S^\ast$ is a reference area,
$c$ is the speed of sound, $Z^\ast$ is a reference acoustic impedance, $T$ is a characteristic time, 
$\alpha$ and $\beta$ are parameters related to the radiation effects, $L$ is the length of the instrument,
$\zeta_\ell(y)$ is a function representing the lip-reed coupling, $F$ is a nonlinear function ($F(\Delta p)=\Delta p \, \text{sign} (\Delta p)$),
$\gamma$ is the dimensionless blowing pressure, $\varepsilon$ =-1 for a reed instrument, 
$\kappa$ is a parameter related to the lip dynamics, $\omega_r$ is the reed's (or lips') frequency ($=2\pi f_r$), 
$Q_r$ is the quality factor of the reed, and $y$ is the dimensionless reed opening. \newline
This kind of system of equation can already be solved numerically with Openwind \cite{OpenwindInria}.

\subsection{Objectives}
The main objectives of this project were:
\begin{enumerate}
    \item Implement a Discrete Galerkin method to numerically solve the mixt form of the wave equation.
    \item Couple the solving with the ODE and the boundary conditions.
    \item To be able to compute the gradient with respect to all the problem parameters.   
\end{enumerate}

\section{Methods}
Describe the method, model or algorithm. Include the essential equations.
\subsection{Discrete Galerkin Method \cite{HesthavenWarburton2008}}

The Discrete Galerkin Method (DGM) is a numerical technique used to solve partial differential equations (PDEs). 
Let's consider an equation of the form :
\begin{equation}
    \frac{\partial u}{\partial t} + \frac{\partial F(u)}{\partial x} = 0,
\end{equation}
where $u(x,t)$ is the unknown function and $F(u)$ is a flux function. \newline
The DGM involves the following steps:
\begin{enumerate}
    \item \textbf{Domain Discretization:} Divide the spatial domain into a finite number of elements or cells.
    \item \textbf{Basis Functions:} Choose a set of basis functions (often polynomials) to represent the solution within each element.
    \item  \textbf{Weak Formulation:} Multiply the PDE by a test function (which can be one of the basis functions) and integrate over each element to obtain the weak form of the equation.
    \item \textbf{Numerical Fluxes:} Define numerical fluxes at the interfaces between elements to ensure stability and consistency.
    \item \textbf{Time Integration:} Use an appropriate time-stepping method (e.g., Runge-Kutta) to advance the solution in time.
\end{enumerate} 

Let's denote the approximate solution at the $n$-th time step as $u^n(x)$. The update from time step $n$ to $n+1$ can be expressed as:
\begin{equation}
    u^{n+1}(x) = u^n(x) - \Delta t \partial_x F(u^n)
\end{equation}

Let's consider test function $\phi_i(x)$, the weak form of the equation becomes in a one cell $\Omega_k$:
\begin{equation}
    \int_{\Omega_k} \phi_i(x)  u^{n+1} \, dx =  \int_{\Omega_k} \phi_i(x)  u^{n} \, dx - \Delta t \int_{\Omega_k} \phi_i(x) \partial _x F(u^n) \, dx
\end{equation}
Using integration by parts on the second term, we get:
\begin{equation}
    \int_{\Omega_k} \phi_i(x)  u^{n+1} \, dx =  \int_{\Omega_k} \phi_i(x)  u^{n} \, dx + \Delta t \int_{\Omega_k} \partial_x \phi_i(x) F(u^n) \, dx - \Delta t \left[ \phi_i(x) F(u^n) \right]_{\partial \Omega_k}
\end{equation}
where $\left[ \phi_i(x) F(u^n) \right]_{\partial \Omega_k}$ represents the numerical flux at the boundaries of the element $\Omega_k$.

Finally, by expressing $u^{n+1} = \sum_j \alpha_{k,j} \phi_j(x)$ within each element, we can derive a system of equations for the coefficients $\alpha_{k,j}$ that can be solved at each time step :
\begin{equation}
    \sum_j \alpha_{k,j}^{n+1} \int_{\Omega_k} \phi_i(x) \phi_j(x) \, dx = \int_{\Omega_k} \phi_i(x)  u^{n} \, dx + \Delta t \int_{\Omega_k} \partial_x \phi_i(x) F(u^n) \, dx - \Delta t \left[ \phi_i(x) F(u^n) \right]_{\partial \Omega_k}
\end{equation}

where $M_{ij} = \int_{\Omega_k} \phi_i(x) \phi_j(x) \, dx$ is a component of the mass matrix $M$.

And so finally we can write the update rule for the coefficients $\alpha_{k,j}$ as:
\begin{equation}
    \alpha_{k}^{n+1} = M^{-1} \left( b \right)
\end{equation}
where $b$ is the right-hand side vector containing the integrals and numerical fluxes.

\subsection{Analitical solving of a simplified model}
Let's condsider the following equation system :
\begin{equation}
\partial_t u +cA\partial_x u = 0
\end{equation}

where $u = \begin{pmatrix} p \\ v \end{pmatrix}$ and $A = \begin{pmatrix} 0 & 1 \\ 1 & 0 \end{pmatrix}$. \newline
and the initial condition : $p(x,0) = p_0(x)$ and $v(x,0) = v_0(x)$ \newline
We obtain the following system 1-D hperbolic system form of the wave equation :

\begin{align}
    \partial_t p + c\,\partial_x v &= 0, \\
    \partial_t v + c\,\partial_x p &= 0.
\end{align}
Introduce the Riemann variables
\begin{equation}
    r := p+v,\qquad s := p-v.
\end{equation}
They satisfy decoupled transport equations
\begin{align}
    \partial_t r + c\,\partial_x r &= 0, \label{eq:transport_r}\\
    \partial_t s - c\,\partial_x s &= 0. \label{eq:transport_s}
\end{align}
With initial data $p(x,0)=p_0(x)$, $v(x,0)=v_0(x)$ we have $r_0=p_0+v_0$ and $s_0=p_0-v_0$, hence from \eqref{eq:transport_r}--\eqref{eq:transport_s}
\begin{equation}
    r(x,t)=r_0(x-ct),\qquad s(x,t)=s_0(x+ct).
\end{equation}
Recovering $(p,v)$ gives the explicit d'Alembert-type solution
\begin{align}
    p(x,t) &= \tfrac12\bigl[r_0(x-ct)+s_0(x+ct)\bigr] \nonumber\\
                 &= \tfrac12\bigl[p_0(x-ct)+p_0(x+ct)\bigr] 
                        + \tfrac12\bigl[v_0(x-ct)-v_0(x+ct)\bigr], \\
    v(x,t) &= \tfrac12\bigl[r_0(x-ct)-s_0(x+ct)\bigr] \nonumber\\
                 &= \tfrac12\bigl[v_0(x-ct)+v_0(x+ct)\bigr] 
                        + \tfrac12\bigl[p_0(x-ct)-p_0(x+ct)\bigr].
\end{align}

\newpage
\printbibliography

% Minimal example of an external bibliography (create refs.bib)
% Example BibTeX entry:
%
% @article{doe2020,
%   author = {Doe, John and Example, Anne},
%   title = {Title of the cited article},
%   journal = {Example Journal},
%   year = {2020},
%   volume = {12},
%   pages = {34--56}
% }

\end{document}